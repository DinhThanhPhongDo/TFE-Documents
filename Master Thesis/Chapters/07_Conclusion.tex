\chapter{Conclusion}
Recent improvements in robot autonomy have opened up possibilities for future applications in the construction industry. One potential application is monitoring changes or faults that may occur in construction sites. The purpose of this master thesis is to investigate potential approaches for future application in the construction industry demonstrating a proof of concept.\\

To begin, a review of the existing literature was conducted to explore relevant methods and frameworks that can serve as our proof of concept. The main frameworks for performing the designated task are presented and explained in this thesis. Two main approaches were identified. The first method is the classical approach which encompasses RANSAC-based algorithms, and geometric properties-based matching. The second approach is a deep learning one, based on PointNet, PointNet++, FlowNet3D.\\

Both methods were implemented and evaluated using an artificial dataset. Although the generated datasets were simplistic, they provided sufficient richness and diversity for the initial proof of concept and training purposes. Evaluation metrics were introduced for both classification and semantic segmentation tasks, and our two models were assessed accordingly.\\

In terms of classification, we achieved an average f1-score of 84.8\%. The deep learning approach exhibited performance ranging from 88.1\% to 97.5\%. For semantic segmentation, the average Intersection over Union (IoU) was 71.0\%, with the classical and deep learning methods achieving IoU values ranging from 35.6\% to 80.3\%.

In conclusion, the obtained results indicate the deep learning approach, being a more adaptable framework, can outperform the classical approach in change detection tasks. These findings highlight the potential for further improvements in the promising deep-learning model.
